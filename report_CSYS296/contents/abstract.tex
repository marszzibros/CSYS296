\begin{abstract}
COVID-19, also known as SARS-CoV-2, is a highly contagious respiratory illness that has had a significant global impact since it was first detected in Wuhan, China, in 2019. Its severe symptoms have posed a threat to public health, and various governmental regulations, such as quarantining, masking, and hand-washing, have been suggested and evaluated to mitigate the threat \cite{haug_2020_ranking}. However, the pandemic's duration has been a matter of public concern as it has persisted for over three years. To predict and estimate the pandemic's end, several methods have been suggested, such as the statistical power series model and the generalized Lokta-Volterra model (gLV). Although the statistical power series model can estimate the number of confirmed infected people, it cannot account for essential factors like immigration or infection rates \cite{baniyounes_2020_covid19}. In their paper, Younes and Hasan (2020) used gLV, an ecological model that simulates the interaction dynamics of multiple species, to simulate the interaction dynamics of COVID-19's populations (healthy and infected). gLV is a widely used ecological model that explains interaction dynamics between more than two species and has been used to analyze gut microbiome communities (Venturelli et al., 2018)\cite{venturelli_2018_deciphering}. Younes and Hasan (2020) attempted to predict COVID-19's behavior from the early stages, and according to their analysis, the infection rate ($\beta$) was critical to the dynamics. The review also implemented control sections not implemented in the original paper. The results suggest that COVID-19 will not be eradicated but will stabilize over time and live within populations.

\vspace{0.95cc}
\parbox{24cc}{{\it COVID-19, Generalized Lotka-Volterra (gLV), Data Analysis, Computational Systems Biology}:%Fill maximum 5 key words
}

\end{abstract}

