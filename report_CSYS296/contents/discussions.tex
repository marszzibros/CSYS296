\section{Discussions} \label{subm}
\hspace{5mm} The gLV model developed by Younes and Hasan (2020) provides a valuable framework for understanding how healthy and infected populations interact in an ecological perspective. While the lack of data on COVID-19 vaccines and treatments prevented the implementation of controls, setting $\epsilon$ to 0 still provided useful insights into how COVID-19 can affect human populations.

For example, the study concluded that the infection rate is a crucial factor for the interaction between healthy and infected populations. Figure 04 and Figure 05, which depict the impact of varying infection rates ($\beta$), further support this finding, as the higher infection rate in Figure 05 (0.05) led to a larger number of infected individuals compared to Figure 04 (0.005). Moreover, Figure 01 implies that if we had regulated travel or attempted to control the spread more strictly earlier, the pandemic might have ended earlier. With controls in Figure 06, it implies that COVID-19 might not disappear in the populations but live with us. As many experts argue from the early stage, the model also expected that infected populations will not diminish.

Compared to the statistical power series model described in section 2.2, the gLV model implemented controlling factors such as immigration and infection rate. However, the model's accuracy in predicting or simulating COVID-19 would have been enhanced if it could have incorporated the infection rates of various COVID-19 variants. To date, over 10 variants of COVID-19 have been identified, each with distinct death and infection rates. Incorporating this data could have further refined the model's predictions and given a more accurate representation of the spread and impact of the virus.

Another limitation of the gLV model is its computational constraints. The model can only simulate an arbitrary country with 100 people due to computational limitations. However, if computation is not an issue, utilizing real numbers from a specific country (such as the U.S.) could enable us to validate the model's accuracy.

One argument I will bring in is that the subject of the research did not necessarily need to be COVID-19. This model can be applied in any contagious disease. Of course, COVID-19 was the biggest pandemic happened in many years, and it is an ideal research topic to make the audience aware of what strategy we need to adopt. However, choosing COVID-19 was purposeful, and I have a feeling that the researchers want to get more attention by mentioning COVID-19. 

The paper by Younes and Hasan (2020) implemented the Extended Kalman Filter (EKF) to study the dynamics of infectious disease transmission. Unfortunately, due to time limitations, the present study was not able to explore this aspect in depth. However, it is worth noting that if EKF had been implemented in Section 2.4, the results might have been different. This is because EKF adds uncertainty to the dynamics of the system, which can have a significant impact on the outcomes. Therefore, future studies that implement EKF could provide a more accurate representation of the spread and impact of infectious diseases such as COVID-19.

Despite these limitations, the gLV model remains a valuable tool for understanding the dynamics of infectious disease transmission and the interactions between healthy and infected populations. With further refinement and incorporation of additional data, the gLV model may provide an even more accurate representation of the spread and impact of infectious diseases like COVID-19. This, in turn, could help inform public health policies and interventions aimed at mitigating the spread of infectious diseases and reducing their impact on human populations.